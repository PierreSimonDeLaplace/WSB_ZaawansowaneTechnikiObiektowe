\section{Refaktoryzacja}

Refaktoryzacja jest procesem zmiany \textbf{działającego} kodu, bez wpływania na jego zachowanie (kod przed i~po refaktoryzacji powinien działać tak samo). Poprawie ulegają jego wewnętrzne struktury. Podczas procesu refaktoryzacji kod powinien być cały czas sprawny. 

Refaktoryzacja powinna być rzeczą naturalną podczas rozwijania oprogramowania. Pozwala ona na zwiększenie czytelności kodu. Czytelny, łatwy do zrozumienia kod jest ważny zarówno dla kolegów z zespołu jak i~dla nas samych wracających po pewnym czasie to kodu nad którym wcześniej pracowaliśmy. Refaktoryzacja polepsza projekt, ułatwia jego modyfikowanie i~rozszerzanie. Dodatkowo sprawia, że znalezienie błędów staje się prostsze, a samo programowanie szybsze i~sprawniejsze.
% Często jak się zaczyna dany projekt to prace nad nim idą jak burza, dopiero to pewnym czasie następuje stagnacja. Refaktoryzacja pozwala uniknąć tej stagnacji, jeśli projekt jest dobrze rozwijany dodawanie nowych funkcjonalności nie powinno być dużym problemem.  

\subsection{,,Zapachy'' w kodzie}
W kontekście refaktoryzacji często pojawia się termin ,,zapach'' kodu. ,,Zapaszki'' sugerują miejsca, gdzie w naszym kodzie może dziać się coś złego. Wskazują na fragmenty, które potencjalnie powinny zostać podane refaktoryzacji. W~książce\cite{refaktoryzacja_fowler} Martin Fowler wyróżnia następujące ,,zapachy'':
\begin{itemize}
	\item Tajemnicza nazwa - nazwy metod, klas, pól powinny jasno określać za co dana składowa jest odpowiedzialna. Dzięki narzędziom refaktoryzacyjnym dostępnym w środowiskach IDE, zmiana nazw jest bardzo prosta i warto z niej często korzystać.
	\item Duplikowany kod - jeśli fragment kodu pojawia się w wielu miejscach jego utrzymanie staje się trudne. Konieczne jest pamiętanie o wprowadzaniu zmian w kilku miejscach, niezbędne jest również czytanie tych samych fragmentów kodu w~różnych plikach.
	\item Długa funkcja - im dłuższa jest metoda tym jej zrozumienie staje się trudniejsze. Warto dekomponować długie metody na kilka mniejszych stosując ekstrakcję funkcji.
	\item Długa lista parametrów - jeśli funkcja ma wiele parametrów trudno jest z niej korzystać. Mimo, że współczesne IDE ułatwiają proces wywoływania metod przez wyświetlanie jej parametrów, to funkcje z mniejszą ich liczbą są czytelniejsze. Zmniejszyć listę parametrów można np. przez zebranie funkcji w~klasę i stosowanie pól klasy, zamiast parametrów. 
	\item Dane globalne - stosowania zmiennych globalnych czy klas (singletonów) należy unikać, nie wiadomo kiedy i gdzie zostaną one zmienione. Błędy z nimi związane są bardzo ciężkie do debugowania. 
	\item Dane mutowalne - problem z danymi zmiennymi jest podobny do tego z~danymi globalnymi. Tam gdzie to możliwe lepiej jest stosować klasy niezmienne. W C\# od wersji 9.0 można korzystać z rekordów.
	\item Rozbieżne zmiany - powstają w skutek nieprzestrzegania zasady SRP\ref{lab1/sec/srpPrinciple}. ,,Zapach'' ten pojawia się jeżeli jedna metoda/klasa jest modyfikowana z różnych powodów, zmian funkcjonalności. Np. zmiana bazy danych czy rozliczeń finansowych pociąga za zmiany kilku metod danej klasy.
	\item Fala uderzeniowa - jest odwrotnością rozbieżnych zmian. Problem powstaje jeżeli jedna konkretna zmiana w~programie pociąga za sobą konieczność wielu innych zmian, w innych fragmentach kodu. Jest ona podobna do rozbieżnych zmian, które odnoszą się do wielu zmian z pojedynczej klasie, natomiast fala uderzeniowa odnosi się do sytuacji kiedy, jedna zmiana generuje zmiany w wielu innych miejscach.
	\item Zazdrosne funkcjonalności - problem pojawia się, gdy dana metoda odwołuje się w większości do metod z innej klasy. Może to oznaczać, że powinna ona być częścią tej klasy. W takiej sytuacji należy przenieść tę funkcję w miejsce, do którego bardzie pasuje.
	\item Stada danych - ,,zapach'' powstaje jeśli pola często występują blisko siebie. Pogrupować je można poprzez ekstrakcję do nowej klasy. 
	\item Opętanie typami prostymi - często warto zamiast stosować proste typu: \texttt{int}, \texttt{double}, \texttt{bool} itd. zamienić je własnymi typami. Np. nr telefonu lepiej jest umieścić w osobnej klasie, zamiast stosować łańcuchy znaków.
	\item Powtarzane instrukcje warunkowe - mogą zostać zastąpione polimorfizmem. Klauzule \texttt{if-else} oraz \texttt{switch-case} często można całkowicie usunąć. W przypadku wielu takich instrukcji warunkowych, może to znacząco poprawić czytelność kodu.
	\item Pętle - zamiast je stosować warto rozważyć użycie potoków. W przypadku C\# dobrym rozwiązaniem jest wykorzystanie wyrażeń LINQ.
	\item Leniwa klasa - ,,zapach'' występuje jeśli klasa nie wykonuje nic ponad delegowanie zadań do innych klas. Rola tej klasy mogła się zmniejszyć np. w~wyniku wcześniejszych refaktoryzacji. 
	\item Spekulacyjne uogólnienia - pojawia się jeśli kod jest nadmiarowy, zaprojektowany na przypadki, które być może kiedyś wystąpią, tak na wszelki wypadek.
	\item Pole tymczasowe - przypadek, w którym pewna wartość jest przypisywana do zmiennej tylko w pewnych sytuacjach. Osoba czytająca kod zwykle oczekuje, że obiekt potrzebuje wszystkich swoich zmiennych.
	\item Łańcuchy komunikatów - pojawia się jeśli klient żąda pewnego obiektu, natomiast ten obiekt żąda innego obiektu, a ten inny obiekt jeszcze innego. Każda zmiana struktury nawigacji wymaga zmian w~kodzie klienta.
	\item Pośrednik - występuje jeśli dana klasa jedyne co robi to deleguje do innych klas. Zamiast korzystać z~tej klasy, często lepiej jest odwołać się bezpośrednio do obiektu, który wie co robić.
	\item Niestosowna bliskość - ma miejsce jeśli dwie klasy wymieniają ze sobą dużą liczbę informacji. Np. jeśli jedna klasa korzysta z klasy danych to być może część tych operacji mogłaby zostać do niej przeniesiona. 
	\item Duża klasa - ,,zapach'' pojawia się jeśli klasa posiada wiele pól, metod czy po prostu linii kodu. Często taka klasa powinna zostać podzielona na kilka mniejszych.
	\item Alternatywne klasy z różnymi interfejsami - występuje, gdy dwie klasy, które mogłyby być stosowane zamienne, posiadają różne interfejsy. 
	\item Klasa danych - posiada jedynie pola i metody dostępowe do nich (gettery i settery). Nie posiadają one żadnego dodatkowego zachowania, a klasy kliencie jedynie manipulują na danych w~nich zawartych. Wyjątkiem od tej zasady jest klasa zwracająca rekord danych.  
	\item Odmowa przyjęcia spadku - ma miejsce, gdy klasy pochodne, korzystają jedynie z części składowych klasy rodzica. Może to świadczyć o źle zaprojektowanej hierarchii klas. Nie zawsze ten ,,zapach'' oznacza coś złego, często można spotkać się z sytuacją w której tworzy się klasę pochodną, aby wykorzystać tylko fragment klasy bazowej. 
\end{itemize}

\subsection{Zadanie 1}

Z \href{https://gitlab.com/PierreSimonDeLaplace/wsb-zaawansowane-techniki-obiektowe-sources}{repozytorium} pobierz materiały do zadania. W folderze \texttt{Lab6} znajdują się dwa projekty. Pierwszy z nich jest zestawem biblioteki .NET Framework, drugi natomiast zawiera projekt z testami NUnit. Podczas tego zadania będziemy przeprowadzać refaktoryzację klasy \texttt{Statement}\ref{lab6/lst/statement}. Metoda \texttt{CreateStatement} wspomnianej klasy pobiera argumenty wejściowe i na ich podstawie przygotowuje raport. Konkretnie będzie to raport generowany w oparciu o dostępne spektakle teatralne oraz faktury. 

\begin{lstlisting}[label=lab6/lst/statement, caption={Klasa statyczna \texttt{Statement}}]
internal static class Statement
{
	public static string CreateStatement(Invoice invoice, Dictionary<string, Play> plays)
	{
		var totalAmount = 0;
		var volumeCredits = 0;
		var results = $"Rachunek dla {invoice.Customer}";
		
		CultureInfo pl = new CultureInfo("pl-PL");
		
		foreach (var perf in invoice.Performances)
		{
			var play = plays[perf.PlayID];
			var thisAmount = 0;
			
			switch (play.Type)
			{
				case "tragedia":
				thisAmount = 40000;
				if (perf.Audience > 30)
				{
					thisAmount += 1000 * (perf.Audience - 30);
				}
				break;
				
				case "komedia":
				thisAmount = 30000;
				if (perf.Audience > 20)
				{
					thisAmount += 10000 + 500 * (perf.Audience - 20);
				}
				thisAmount += 300 * perf.Audience;
				break;
				
				default:
				throw new ArgumentException($"Nieznany typ przedstawienia: ${play.Type}");
			}
			
			// Award bonus points
			volumeCredits += Math.Max(perf.Audience - 30, 0);
			// Award an additional promotional point for every 5 viewers of the comedy
			if ("komedia" == play.Type)
			{
				volumeCredits += (int)Math.Floor((decimal)perf.Audience / 5);
			}
			
			// Create statement row
			results += $"{play.Name}: {(thisAmount / 100).ToString("c", pl)} (liczba miejsc: {perf.Audience}" + Environment.NewLine;
			totalAmount += thisAmount;
		}
		
		results += $"Naleznosc: {(totalAmount / 100).ToString("c", pl)}" + Environment.NewLine;
		results += $"Punkty promocyjne: {volumeCredits}";
		
		return results;
	}
}
\end{lstlisting}


W projekcie z folderze \texttt{DTOs} znajdują się również klasy wykorzystywane do deserializacji danych typu JSON z plików wejściowych. Konkretnie dla modeli  spektakli (ang. plays) oraz faktur (ang. invoces).
\begin{lstlisting}[caption={Performance.cs}]
public class Performance
{
	[JsonProperty("playID")]
	public string PlayID { get; set; }
	[JsonProperty("audience")]
	public int Audience { get; set; }
}
\end{lstlisting}
\begin{lstlisting}[caption={Invoice.cs}]
public class Invoice
{
	[JsonProperty("customer")]
	public string Customer { get; set; }
	[JsonProperty("performance")]
	public List<Performance> Performance { get; set; }
}
\end{lstlisting}
oraz
\begin{lstlisting}[caption={Play.cs}]
public class Play
{
	[JsonProperty("name")]
	public string Name { get; set; }
	[JsonProperty("type")]
	public string Type { get; set; }
}
\end{lstlisting}


Ponieważ podczas refaktoryzacji zastanego kodu, z którym nie mieliśmy wcześniej doświadczenia istnieje duże ryzyko wprowadzenia nowego błędu, dobrze jest zacząć proces refaktoryzacji od napisania porządnego zestawu testów. W celu maksymalnego uproszczenia tego zadania, w projekcie z testami został dodany tylko jeden test korzystający z plików wejściowych oraz oczekiwanego wyniku, również w takiej formie - w rzeczywistości powinniśmy przygotować zdecydowanie więcej plików, dla różnych sytuacji. Lepiej byłoby przygotować pełne testy refaktoryzowanej klasy, bez korzystania z plików wejściowych, ale pokazane podejście pozwoli na sprawniejsze przeprowadzenie ćwiczenia. W folderze \texttt{GoldenFiles} zostały umieszczone pliki JSON zawierające listę spektakli oraz przykładowe ,,dane do'' faktury.
\begin{lstlisting}[caption={plays.json}]
{
	"hamlet": {
		"Name": "Hamlet",
		"Type": "tragedia"
	},
	"as-like": {
		"Name": "Jak wam sie podoba",
		"Type": "komedia"
	},
	"othello": {
		"Name": "Otello",
		"Type": "tragedia"
	}
}
\end{lstlisting}
oraz
\begin{lstlisting}[caption={invoices.json}]
{
	"customer": "BigCo",
	"performance": [
	{
		"playID": "hamlet",
		"audience": 55
	},
	{
		"playID": "as-like",
		"audience": 35
	},
	{
		"playID": "othello",
		"audience": 40
	}
	]
}
\end{lstlisting}

W jedynej klasie w projekcie z testami zwróć uwagę na metodę testującą. W pierwszej kolejności następuje deserializacja plików JSON na obiekty klas \texttt{Plays} oraz \texttt{Invoice}. Do deserializacji został wykorzystany pakiet \texttt{Newtonsoft.Json}. Dalej do zmiennej \texttt{expected} przypisywany jest obiekt typu \texttt{string}, który zawiera oczekiwany wynik działania testowanej metody. Następnie jest wywoływana wspomniana metoda oraz porównywane są wyniki. Uruchom testy, powinny zakończyć się one wynikiem pozytywnym.

Pora teraz na przyjrzenie się klasie, którą należy w tym zadaniu poprawić. Na pewno klasa łamie zasadę pojedynczej odpowiedzialności\ref{lab1/sec/srpPrinciple} oraz jednocześnie jest dość długa, co pogarsza jej czytelność. Metoda zarówno przygotowuje raport, jak jest odpowiedzialna za naliczania rabatów czy liczenie kosztów całych przedstawień. Sama w sobie nie jest źle napisana, ale na pewno można w nie sporo zmienić.

W pierwszym kroku dokonaj ekstrakcji wyrażenia \texttt{switch} zaznaczając całe wyrażenie i wciskając \texttt{Alt+Enter} oraz wybierając opcję \texttt{Extract method}, nazwij nową metodę \texttt{AmountFor}. Zamiast tworzyć zmienną \texttt{thisAmount} i przypisywać jej najpierw wartość zero, przypisz jej od razu wartość zwracaną przez \texttt{AmountFor}. Zmień za pomocą \texttt{Ctrl+R+R} nazwę zmiennej \texttt{thisAmount} wewnątrz utworzonej przed chwilą metody \texttt{AmountFor} na nazwę lepiej ją opisującą, informującą że jest to wartość zwracana przez tę funkcję np. \texttt{result}. Nazwa parametru \texttt{perf} również nie jest najszczęśliwsza, analogicznie zmień ją na bardziej opisową. Uruchom testy, aby sprawdzić czy nie zostało nic zepsute w między czasie.

Wewnątrz klasy \texttt{Statement} dodaj prywatną metodę \texttt{GetPlay}:
\begin{lstlisting}[caption={Dodana metoda \texttt{GetPlay}}]
Play GetPlay(Dictionary<string, Play> plays, Performance perf) => plays[perf.PlayID];
\end{lstlisting}

Następnie usuń zmienną \texttt{play} wykonując ,,Zastąpienie Zmiennej Tymczasowej Zapytaniem''. Zmień wszystkie wykorzystania zmiennej \texttt{play} na wywołanie metody \texttt{GetPlay(plays, perf)} wykonując ,,Wchłonięcie Zmiennej''. Usuń zmienną \texttt{play} z \textbf{wywołania} funkcji \texttt{AmountFor}, zamień ją odpytaniem metody \texttt{GetPlay()}. Uruchom testy jednostkowe i sprawdź czy wykonują się one poprawnie.


Refaktoryzację ,,Wchłonięcie zmiennej'' wykonaj analogicznie dla \texttt{thisAmount}, zamień ją wywołaniem prywatnej metody \texttt{AmountFor}. Usuń zmienną \texttt{thisAmount}.


Wyodrębnij fragment kodu odpowiedzialny za liczenie rabatu do osobnej metody:
\begin{lstlisting}
private static int VolumeCreditsFor(Dictionary<string, Play> plays, Performance perf)
{
	// Award bonus points
	var volumeCredits = Math.Max(perf.Audience - 30, 0);
	// Award an additional promotional point for every 5 viewers of the comedy
	if ("komedia" == GetPlay(plays, perf).Type)
	{
		volumeCredits += (int)Math.Floor((decimal)perf.Audience / 5);
	}
	
	return volumeCredits;
}
\end{lstlisting}
Dokonaj zmian w pętli \texttt{foreach} tak, aby do zmiennej \texttt{volumeCredits} była dodawana wartość zwracana przez funkcję \texttt{VolumeCreditsFor}. Zmień nazwy zmiennych wewnątrz funkcji \texttt{VolumeCreditsFor} na bardziej opisowe. 


Analogicznie wyodrębnij funkcję formatującą waluty. Utwórz folder \texttt{Extensions} i dodaj do niego klasę statyczną \texttt{CurrencyExtensions}. Następnie dodaj do tej klasy metodę formatującą ciąg znaków. Możesz skorzystać z metody rozszerzającej, która została pokazana w rozdziale \ref{lab3/sec/FacadeUml}.
\begin{lstlisting}
namespace RefactorExample.Extensions
{
	internal static class CurrencyExtensions
	{
		internal static string ToPolishZloty(this int value) => (value / 100).ToString("c", new CultureInfo("pl-PL"));
	}
}
\end{lstlisting}
Zmień kod w miejscach, gdzie były formatowane waluty, korzystając z metody rozszerzającej. Usuń zmienną lokalną \texttt{pl} typu \texttt{CultureInfo}. 


Teraz, aby wchłonąć zmienną \texttt{volumeCredits} zastosuj ,,Podział pętli'' - rozdziel jedną pętle na dwie odpowiedzialne odpowiednio za wyliczanie ceny przedstawienia oraz punktów rabatowych. Możesz również zmienić nazwę zmiennej w~pętli \texttt{foreach}.
\begin{lstlisting}
static string CreateStatement(Invoice invoice, Dictionary<string, Play> plays)
{
	//...	
		
	foreach (var performance in invoice.Performances)
	{
		volumeCredits += VolumeCreditsFor(plays, performance);
	}
	
	foreach (var performance in invoice.Performances)
	{
		// Create statement row
		results += $"{GetPlay(plays, performance).Name}: {AmountFor(performance, GetPlay(plays, performance)).ToPolishZloty()} (liczba miejsc: {performance.Audience}" + Environment.NewLine;
		totalAmount += AmountFor(performance, GetPlay(plays, performance));
	}
	
	//...
}
\end{lstlisting}
Następnie przenieś pętle w której wykonywane jest liczenie rabatu do osobnej metody prywatnej o nazwie \texttt{TotalVolumeCredits} wraz ze zmienną lokalną \texttt{volumeCredits}. Teraz usuń z funkcji \texttt{CreateStatement} zmienną \texttt{volumeCredits} i w miejscu jej użycia umieść wywołanie utworzonej przed chwilą metody. 

Wykonaj analogiczną refaktoryzację ze zmienną \texttt{totalAmount} - nowo tworzoną metodę nazwij \texttt{TotalAmount}. Wcześniej zastosuj ponownie podział pętli na jedną aktualizującą \texttt{results} oraz tę aktualizującą \texttt{totalAmount}.
\begin{lstlisting}
public static string CreateStatement(Invoice invoice, Dictionary<string, Play> plays)
{
	var results = $"Rachunek dla {invoice.Customer}";
	
	foreach (var performance in invoice.Performances)
	{
		// Create statement row
		results += $"{GetPlay(plays, performance).Name}: {AmountFor(performance, GetPlay(plays, performance)).ToPolishZloty()} (liczba miejsc: {performance.Audience}" + Environment.NewLine;
	}
	
	results += $"Naleznosc: {TotalAmount(invoice, plays).ToPolishZloty()}" + Environment.NewLine;
	results += $"Punkty promocyjne: {TotalVolumeCredits(invoice, plays)}";
	
	return results;
}

private static int TotalAmount(Invoice invoice, Dictionary<string, Play> plays)
{
	var totalAmount = 0;
	
	foreach (var performance in invoice.Performances)
	{
		totalAmount += AmountFor(performance, GetPlay(plays, performance));
	}
	
	return totalAmount;
}
\end{lstlisting}


Wykonując powyższe kroki udało się poprawić czytelność kodu, ale jego funkcjonalność może zostać ulepszona. Jeśli chcielibyśmy np. zmienić funkcję, aby formatowała oświadczenie do formatu HTML, konieczne byłoby przekopiowanie dużej części kodu. Spróbujmy rozwiązać ten problem.

Przenieś cały kod metody \texttt{CreateStatement} do osobnej metody (wykonaj ,,Ekstrakcję Funkcji'') o nazwie \texttt{RenderPlainText}. Teraz utwórz w projekcie folder \texttt{Statements}, a nim klasę \texttt{StatementData}, która będzie służyła za pośrednika danych między metodami \texttt{CreateStatement} oraz \texttt{RenderPlainText}. Ta refaktoryzacja nazywa jest ,,Wprowadzenie Obiektu Parametrycznego''. W metodzie \texttt{CreateStatement} utwórz obiekt typu \texttt{StatementData} i przekaż go do wnętrza metody \texttt{RenderPlainText}:
\begin{lstlisting}
static string CreateStatement(Invoice invoice, Dictionary<string, Play> plays)
{
	StatementData statementData = new StatementData();
	
	return RenderPlainText(statementData, invoice, plays);
}

private static string RenderPlainText(StatementData data, Invoice invoice, Dictionary<string, Play> plays)
{
	var results = $"Rachunek dla {invoice.Customer}";
		
	//...
}
\end{lstlisting}

Do klasy \texttt{StatementData} dodaj publiczne właściwości \texttt{List<Performance> Performances} oraz \texttt{string Customer}. Możesz, też dodać prywatne pole, a kolekcję obiektów \texttt{Performance} zwracać jako kolekcję tylko do odczytu, tak jak pokazano to na~\ref{lab6/lst/statementData}.Teraz w~funkcji \texttt{CreateStatement} zmień wywołanie konstruktora i przekaż do niego odpowiednie wartości z właściwości obiektu \texttt{invoice}. Wszystkie użycia \texttt{invoice} w metodzie \texttt{RenderPlainText} zamień użyciem nowego obiektu pośredniczącego. Z sygnatury metody \texttt{RenderPlainText} usuń parametr typu \texttt{Invoice}, od teraz wszystkie potrzebne dane będą przekazywane za pomocą obiektu pośredniego.
\begin{lstlisting}[label=lab6/lst/statementData]
private class StatementData
{
	public string Customer { get; }
	
	private List<Performance> performances;
	
	public StatementData(string customer, List<Performance> performances)
	{
		this.Customer = customer;
		this.performances = performances;
	}
	
	public IEnumerable<Performance> Performances => this.performances.AsReadOnly();
}

static string CreateStatement(Invoice invoice, Dictionary<string, Play> plays)
{
	StatementData data = new StatementData(invoice.Customer, invoice.Performances);
	
	return RenderPlainText(data, plays);
	
}

private static string RenderPlainText(StatementData data, Dictionary<string, Play> plays)
{
	var results = $"Rachunek dla {data.Customer}";
	
	//...
}
\end{lstlisting}
Konieczna okaże się również zmiana sygnatury funkcji \texttt{TotalVolumeCredtis} oraz \texttt{TotalAmount}. Zamiast przyjmować argument typu \texttt{Invoice} niech przyjmują argumenty typu \texttt{IEnumerable<Performance>}.


Klasa \texttt{Performance} służyła do deserializacji pliku JSON. Wydaje się dobrym pomysłem, aby dodać więcej informacji do klasy opisującej przedstawienia np. obiekt typu \texttt{Play}, cenę za przedstawienie oraz punkty rabatowe. Dodaj do projektu folder \texttt{Entities}, a w nim umieść klasę \texttt{EnrichedPerformance} w osobnym pliku:
\begin{lstlisting}
namespace RefactorExample.Entities
{
	public class EnrichedPerformance
	{
		public EnrichedPerformance(string playId, int audience, int amount, Play play, int volumeCredits)
		{
			PlayId = playId;
			Audience = audience;
			Amount = amount;
			Play = play;
			VolumeCredits = volumeCredits;
		}
		
		public string PlayId { get; }	
		public int Audience { get; }	
		public int Amount { get; }
		public Play Play { get; }
		public int VolumeCredits { get; }
	}
}
\end{lstlisting}

Zanim wykonamy następne kroki, najwyższa pora usunąć słowo kluczowe \texttt{static} znajdujące się przed klasą \texttt{Statement}. Po jego usunięciu, dodaj do klasy konstruktor przyjmujący dwa argumenty: jeden typu \texttt{Invoice} drugi natomiast typu \texttt{Dictionary<string, Play> plays}. Oba przekazywane argumenty przypisz do prywatnych pól:
\begin{lstlisting}
internal class Statement
{
	private readonly Invoice invoice;
	private readonly Dictionary<string, Play> plays;
	
	public Statement(Invoice invoice, Dictionary<string, Play> plays)
	{
		this.invoice = invoice;
		this.plays = plays;
	}

	//...
}
\end{lstlisting}

Teraz usuń słowo kluczowe \texttt{static} z sygnatury metody \texttt{CreateStatement} oraz usuń oba jej parametry. Od tego momentu można będzie posługiwać się obiektami przekazywanymi jako argumenty konstruktora. Konieczna okaże się również zmiana w projekcie \texttt{RefactorExample.Tests}. Zmień ciało metody zawierającej testy:
\begin{lstlisting}
[Test]
public void ShouldGenerateStatementIfValidInput()
{
	//Arrange
	var inputPlays = JsonConvert.DeserializeObject<Dictionary<string, Play>>(File.ReadAllText(Path.Combine(baseDirectory, "GoldenFiles\\inputPlays.json")));
	var inputInvoice = JsonConvert.DeserializeObject<Invoice>(File.ReadAllText(Path.Combine(baseDirectory, "GoldenFiles\\inputInvoices.json")));
	var _sut = new Statement(inputInvoice, inputPlays);
	
	var expected = File.ReadAllText(Path.Combine(baseDirectory, "GoldenFiles\\expectedOutput.txt"));
	
	//Act
	var actual = _sut.CreateStatement();
	
	//Assert
	Assert.AreEqual(expected, actual);
}
\end{lstlisting}
Uruchom ponownie testy, powinny się one zakończyć ,,na zielono''.


Następnie dodaj do klasy \texttt{Statement} metodę, która będzie odpowiedzialna za mapowanie obiektów \texttt{Performance} na \texttt{EnrichedPerformance}:
\begin{lstlisting}
public EnrichedPerformance EnrichPerformance(Performance performance)
{
	return new EnrichedPerformance(
		performance.PlayID,
		performance.Audience,
		AmountFor(performance, GetPlay(plays, performance)),
		GetPlay(plays, performance),
		VolumeCreditsFor(plays, performance));
}
\end{lstlisting}



Zamień typ przechowywany w liście \texttt{Performances} klasy \texttt{StatementData} na \texttt{EnrichedPerformance}. Podczas tworzenia obiektu typy \texttt{Statement} w metodzie \texttt{StatementData} zastosuj napisaną przed chwilą metodę w celu zmapowania obiektu \texttt{Performance} na \texttt{EnrichedPerformance}:
\begin{lstlisting}
public string CreateStatement()
{
	StatementData statementData = new StatementData(invoice.Customer, 
		invoice.Performances.Select(x => EnrichPerformance(x)).ToList());
	
	return RenderPlainText(statementData, plays);
}
\end{lstlisting}
Zamień sygnatury metod: \texttt{TotalAmount} oraz \texttt{TotalVolumeCredtis} tak, aby przyjmowały obiekty typu \texttt{EnrichedPerformance}. W metodach \texttt{RenderPlainText}, \texttt{TotalAmount} oraz \texttt{TotalVolumeCredits} pojawiły się błędy. Popraw je wykorzystując właściwości \texttt{VolumeCredits} oraz \texttt{Amount} kolekcji \texttt{Performances} obiektu \texttt{StatementData}.




W metodzie \texttt{EnrichPerformance} dodaj przypisanie wartości kolejnej właściwości:
\begin{lstlisting}
private EnrichedPerformance EnrichPerformance(Performance performance)
{
	return new EnrichedPerformance(
		performance.PlayID,
		performance.Audience,
		AmountFor(performance, GetPlay(plays, performance)),
		GetPlay(plays, performance),
		VolumeCreditsFor(plays, performance));
}
\end{lstlisting}
Zwróć uwagę, że z funkcji \texttt{TotalAmount} oraz \texttt{TotalVolumeCredtis} możesz usunąć parametr \texttt{Dictionary<string, Play> plays}. Popraw błędy w metodzie \texttt{RenderPlainText}. 


Dalej usuń słowo kluczowe \texttt{static} ze wszystkich statycznych metod w klasie \texttt{Statement}. Usuń z ich sygnatur argument typu \texttt{Dictionary<string, Play> plays}. Ponownie popraw błędy, które się pojawiły i~uruchom jeszcze raz testy.


Następnie wydaje się możliwe, aby przenieść metody \texttt{TotalAmount} oraz \texttt{TotalVolumeCredtis} do klasy \texttt{StatementData}, jednocześnie usuwając ich jedyny parametr \texttt{IEnumerable<EnrichedPerformance> performances}. Możesz również pętle w~metodach \texttt{TotalAmount} oraz \texttt{TotalVolumeCredits} zamienić na potok:
\begin{lstlisting}
private int TotalAmount(StatementData statementData)
{
	var totalAmount = 0;
    performances.ForEach(x => totalAmount += x.Amount);

	
	return totalAmount;
}
\end{lstlisting}
Ponownie popraw błędy, które się pojawiły i uruchom testy.


Do projektu dodaj folder \texttt{Render}, a w nim klasę \texttt{PlainTextRenderer}. Przenieś do niej metodę \texttt{RenderPlainText} z klasy \texttt{Statement}, zmień jej nazwę na \texttt{Render} oraz modyfikator dostępu z \texttt{private} na \texttt{public}. Metodę \texttt{CreateStatement} zmień w następujący sposób:
\begin{lstlisting}
public string CreateStatement()
{
	StatementData statementData = new StatementData(invoice.Customer, 
		invoice.Performances.Select(x => EnrichPerformance(x)).ToList());
	
	return new PlainTextRenderer().Render(statementData);
}
\end{lstlisting}
Możesz również utworzyć w~folderze \texttt{Render} interfejs \texttt{IRenderer}, który będzie implementowany przez klasę \texttt{PlainTextRenderer}:
\begin{lstlisting}
public interface IRenderer
{
	string Render(StatementData data);
}
\end{lstlisting}


Na sam koniec spróbujmy pozbyć się instrukcji \texttt{switch} z metod \texttt{AmountFor} oraz \texttt{VolumeCreditsFor}. Najpierw utwórz folder \texttt{Calculators}. W nim natomiast interfejs \texttt{IPerformanceCalculator}:
\begin{lstlisting}
interface IPerformanceCalculator
{
	int AmountFor(int audience);
	int VolumeCreditsFor(int audience);
}
\end{lstlisting}
Dalej do folderu dodaj dwie klasy \texttt{ComedyCalculator} oraz \texttt{TragedyCalculator}. Obie niech implementują utworzony przed chwilą interfejs. Następnie przenieś kod z metod \texttt{AmountFor} oraz \texttt{VolumeCreditsFor} do odpowiednich klas:
\begin{lstlisting}
class TragedyCalculator : IPerformanceCalculator
{
	public int AmountFor(int audience)
	{
		int result = 40000;
		if (audience > 30)
		{
			result += 1000 * (audience - 30);
		}
		
		return result;
	}
	
	public int VolumeCreditsFor(int audience)
	{
		return Math.Max(audience - 30, 0);
	}
}
\end{lstlisting}
oraz 
\begin{lstlisting}
class ComedyCalculator : IPerformanceCalculator
{
	public int AmountFor(int audience)
	{
		int result = 30000;
		if (audience > 20)
		{
			result += 10000 + 500 * (audience - 20);
		}
		result += 300 * audience;
		
		return result;
	}
	
	public int VolumeCreditsFor(int audience)
	{
		var volumeCredits = Math.Max(audience - 30, 0);
		volumeCredits += (int)Math.Floor((decimal)audience / 5);
		
		return volumeCredits;
	}
}
\end{lstlisting}

Dalej dodaj do folderu \texttt{Calculators} klasę fabryki z metodą wytwórczą w~następujący sposób:
\begin{lstlisting}
class FactoryCalculator
{
	public static IPerformanceCalculator Create(Play play)
	{
		switch(play.Type.ToLowerInvariant())
		{
			case "tragedia":
				return new TragedyCalculator();
			case "komedia":
				return new ComedyCalculator();
			default:
				throw new ArgumentException($"Unknown type of play");
		}
	}
}
\end{lstlisting}

Tylko w jednym miejscu - w klasie fabryki - będzie tworzona odpowiednia instancja klasy, dalej będziemy korzystać z~mechanizmu polimorfizmu. Skorzystaj z dodanych klas w metodzie \texttt{EnrichPerformance}:
\begin{lstlisting}
private EnrichedPerformance EnrichPerformance(Performance performance)
{
	IPerformanceCalculator calculator = FactoryCalculator.Create(GetPlay(performance));
	
	return new EnrichedPerformance(
		performance.PlayID,
		performance.Audience,
		calculator.AmountFor(performance.Audience),
		GetPlay(performance),
		calculator.VolumeCreditsFor(performance.Audience));
}
\end{lstlisting}

Omawiany przykład można by zapewne refaktoryzować dalej, być może w trakcie prac nad kodem okazałoby się konieczne ,,pójście'' w inna stronę. Tym nie mniej z dość chaotycznej pojedynczej metody udało nam się wprowadzić strukturę, która dobrze tłumaczy kod i pozwala na jego rozszerzanie. Jeśli przyjdzie potrzeba zmienić sposób renderowania rachunku to wystarczy dodać nową klasę implementującą \texttt{IRenderer} i ją wykorzystać. Jeśli pojawi się potrzeba dodania nowego sposobu wyliczania rabatów wystarczy dodać nowy kalkulator.