\section{Wzorce projektowe (1) - konstrukcyjne}
Wzorce kreacyjne pozwalają na tworzenie obiektów w sposób zwiększający elastyczność. Zamiast tworzyć obiekty w wielu miejscach w programie, zadaniem konstrukcji nowych instancji można obciążyć osobne klasy. Użycie słowa kluczowego \texttt{new} narusza omawianą wcześniej zasadę odwracania zależności, powodując trudności z testowaniem kodu i zmniejszając jego elastyczność. 
%\url{https://softwareengineering.stackexchange.com/questions/365119/is-using-new-in-the-constructor-always-bad}

\subsection{Fabryka abstrakcyjna (ang. abstract factory)}\label{sec/lab2/abstractFactory}
Fabryka abstrakcyjna jest wzorcem projektowym, który przenosi proces tworzenia obiektów (często ze sobą w~jakiś sposób powiązanych) do osobnej klasy implementującej interfejs zawierający metody zwracające te obiekty. Tworzone przez fabryki instancje powinny posiadać zgodne interfejsy.

Klient fabryki może posiadać konstruktor, który jako parametr przyjmie obiekt konkretnej fabryki. Wewnątrz konstruktora klient może przypisać przekazaną instancję do prywatnego pola. Zakładamy, że zarówno prywatne pole wewnątrz klasy klienta jak i typ przekazywanego przez konstruktor argumentu jest typem interfejsu, który implementuje każda konkretna fabryka. Wywołując odpowiednie metody obiektu fabryki, klient może tworzyć konkretne instancje zwracane przez fabrykę bez znajomości ich szczegółowej implementacji. Przekazując inny obiekt fabryki abstrakcyjnej możliwa jest łatwa zmiana pośrednio tworzonych przez klienta obiektów.

Jako przykład użycia omawianego wzorca można wyobrazić sobie aplikację wykorzystującą interfejs graficzny. Klasa tworząca kontrolki interfejsu użytkownika może przyjmować obiekt fabryki implementującej interfejs \texttt{IGuiFactory}. Interfejs ten mógłby definiować metody tworzenia poszczególnych elementów graficznych np. \texttt{CreateButton()}, \texttt{CreateListBox()}, \texttt{CreateProgressBar()} itp. Następnie w zależności od systemu operacyjnego można by stworzyć obiekty fabryk \texttt{WindowsFactory} oraz \texttt{UnixFactory}, oba implementujące określony wcześniej interfejs. Obiekt zawierający implementację logiki rysującej okno użytkownika nie musiałby wiedzieć z jakim systemem operacyjnym ma do czynienia, nie posiadałby zależności do konkretnej implementacji. Mógłby wywoływać metody tworzące kontrolki UI za pomocą abstrakcyjnego interfejsu \texttt{IGuiFactory}. Wybór konkretnej fabryki mógłby być określany w~momencie uruchamiania aplikacji.

\begin{figure}[hbt!]
	\centering
	\includegraphics[width=0.9\linewidth]{images/AbstractFactoryUml}
	\caption{Diagram UML wzorca Fabryka Abstrakcyjna.}
	\label{lab2/fig/AbstractFactoryUml}
\end{figure}
%[Client]->[IAbstractFactory]
%[Client]-[note: ProductA pa = factory.CreateProductA(){bg:cornsilk}]
%[IAbstractFactory]^-.-[ConcreteFactoryA]
%[IAbstractFactory]^-.-[ConcreteFactoryB]
%
%[≪interface≫;IAbstractFactory|+CreataProductA(): ProductA;+CreataProductB(): ProductB;]
%[Client|-factory: IAbstractFactory|+Client(f:AbstractFactory);]
%[ConcreteFactoryA|...|CreateProductA():ProductA;CreateProductB():ProductB]
%[ConcreteFactoryB|...|CreateProductA():ProductA;CreateProductB():ProductB]

%//TODO: https://github.com/ninject/Ninject.Extensions.Factory vs https://docs.autofac.org/en/latest/advanced/delegate-factories.html

\subsubsection{Zadanie 1}

Utwórz nowy projekt aplikacji konsolowej .NET 5.0. W osobnych plikach dodaj do projektu trzy interfejsy: \texttt{IButton}, \texttt{ICheckbox} oraz \texttt{IGuiFactory}. Pierwsze dwa niech posiadają jedną metodę \texttt{void Draw()}, natomiast interfejs fabryki \texttt{IGuiFactory} dwie metody:\texttt{IButton CreateButton()} oraz \texttt{ICheckbox CreateCheckBox()} Typem zwracanym przez te dwie metody są obiekty implementujące odpowiednio \texttt{IButton} oraz \texttt{ICheckbox}.

Umieść w~projekcie dwa foldery \texttt{Macintosh} oraz \texttt{Windows}. Wewnątrz tych folderów dodaj implementacje stworzonych przed chwilą interfejsów. Wszystkie metody \texttt{Draw} zaimplementuj tak, aby wypisywały na ekranie konsoli komunikat symulujący, że kontrolki zostały narysowane np.:
\begin{lstlisting}
internal class MacButton : IButton
{
	public void Draw() => Console.WriteLine("Macintosh button has been drawn");
}
\end{lstlisting}

%Tip: Ustawiając kursor na nazwę interfejsu i klikając Alt+Enter możesz uruchomić narzędzie refaktoryzacyjne i wybrać opcję \texttt{Implement interface}. Visual Studio automatycznie doda do klasy metody wymagające implementacji.

Po napisaniu analogicznych metod dla pozostałych kontrolek, utwórz po jednej klasie fabryki w~każdym z~folderów. Każda klasa fabryki powinna zwracać obiekty konkretnych kontrolek UI np.:
\begin{lstlisting}
public class MacFactory : IGuiFactory
{
	public IButton CreateButton() => new MacButton();
	public ICheckbox CreateCheckBox() => new MacCheckbox();
}
\end{lstlisting}

Teraz w katalogu głównym projektu utwórz klasę \texttt{UserInterface}, która będzie posiadała prywatne pole typu \texttt{IGuiFactory}. Dodaj do tej klasy konstruktor z jednym parametrem typu \texttt{IGuiFactory}. Przypisz przekazywany do wnętrza klasy obiekt do utworzonego przed chwilą prywatnego pola. Napisz metodę \texttt{DrawWindow}, w której pobierzesz z~fabryki instancje klas implementujących \texttt{IButton} oraz \texttt{ICheckbox} i wywołasz metody \texttt{Draw} każdego z~nich.

W metodzie \texttt{Main} sprawdź działanie wzorca fabryki abstrakcyjnej:
\begin{lstlisting}
class Program
{
	static void Main(string[] args)
	{
		// Based on configuration file one can choose the specific factory over another one
		var factoryA = new MacFactory();
		var factoryB = new WinFactory();
		
		var uiA = new UserInterface(factoryA);
		uiA.DrawWindow();
		
		var uiB = new UserInterface(factoryB);
		uiB.DrawWindow();
	}
}
\end{lstlisting}

\subsubsection{Inne wzorce fabryk}
%https://refactoring.guru/design-patterns/factory-comparison
Fabryka jest ogólnym terminem, który jest używany w stosunku do metod i klas, które tworzą obiekty. Podczas laboratoriów został omówiony wzorzec fabryki abstrakcyjnej, który służy do tworzenia całych rodzin podobnych albo powiązanych ze sobą obiektów bez precyzowania konkretnych klas. Jeśli jednak program nie wymaga tworzenia takich rodzin najprawdopodobniej fabryka abstrakcyjna nie jest konieczna. 

Inną prostą fabryką może być klasa, która posiada metodę wytwórczą, z wyrażeniem \texttt{switch-case} które analizuje przekazywany do niej parametr i tworzy odpowiednią instancję konkretnej klasy. Często jest to pojedyncza metoda w~pojedynczej klasie. Z czasem kiedy liczba tworzonych obiektów rośnie może ona się przekształcić z metodę fabrykującą. 
%\begin{lstlisting}
%class UserFactory {
%	public static function create($type) {
%		switch ($type) {
%			case 'user': return new User();
%			case 'customer': return new Customer();
%			case 'admin': return new Admin();
%			default:
%			throw new Exception('Wrong user type passed.');
%		}
%	}
%}
%\end{lstlisting}

Metoda fabrykująca udostępnia interfejs do tworzenia obiektów, ale pozwala klasom pochodnym zmienić typ tworzonego obiektu. W klasie bazowej umieszczona jest pewna funkcjonalność, do klas pochodnych jest przeniesiony jedynie proces tworzenia obiektów. Jest to szczególny przypadek metody szablonowej.
%\begin{lstlisting}
%abstract class Department {
%	public abstract function createEmployee($id);
%	
%	public function fire($id) {
%		$employee = $this->createEmployee($id);
%		$employee->paySalary();
%		$employee->dismiss();
%	}
%}
%
%class ITDepartment extends Department {
%	public function createEmployee($id) {
%		return new Programmer($id);
%	}
%}
%
%class AccountingDepartment extends Department {
%	public function createEmployee($id) {
%		return new Accountant($id);
%	}
%}	
%\end{lstlisting}

\subsection{Budowniczny (ang. builder)}\label{lab2/sec/builderPattern}
Budowniczy jest wzorcem, który może być wykorzystany do tworzenia złożonego obiektu \textbf{krok po kroku}. W przeciwieństwie do wzorca fabryki abstrakcyjnej, która tworzy obiekty \textbf{w~jednym kroku}.

Czasami istnieje pokusa stworzenia konstruktora przyjmującego dużą liczbę argumentów. Może tak się zdarzyć jeśli klasa korzysta z~wielu zależności albo gdy niektóre jej cechy mogą być parametryzowane. Część z tych argumentów może dodatkowo być opcjonalna. Budowniczy przenosi proces tworzenia takiej instancji do osobnego obiektu. 

Czynności wywoływania poszczególnych metod konfigurujących tworzony obiekt można przenieść do obiektu kierownika (ang. director), który będzie przyjmował jako argument konstruktora obiekt budowniczego i~wywoływał metody interfejsu \texttt{IBuilder}. Obiekt kierownika jest opcjonalny, klient może samemu tworzyć instancję budowanego typu. 

Konkretni budowniczowie powinni dostarczyć swoje własne metody zwracania wyników procesu budowania obiektu. Różni budowniczowie mogą zwracać zupełnie inne typy dlatego tego procesu często nie da się umieścić w interfejsie \texttt{IBuilder} (przynajmniej w przypadku języków programowania, które są silnie typowane). 
 
Dodatkowo należy wspomnieć, że wzorzec budowniczy często wykorzystuje tzw. Płynny Interfejs (ang. Fluent Interface). Polega on na łańcuchowym połączeniu metod. Pozwala na zwiększenie czytelności kodu. Kaskadowe połączenie metod realizuje się zwracając w metodzie instancję własnej klasy za pomocą słowa kluczowego \texttt{this}. Z przykładem użycia tego mechanizmu można się spotkać w zapytaniach LINQ\ref{lab2/lst/fluentInterfaceLinq}, gdzie zamiast wywoływać kolejne metody progresywnie (w kolejnych wierszach), można je wywoływać kaskadowo po symbolu kropki. 

\begin{lstlisting}[caption={Wykorzystanie płynnych interfejsów w zapytania LINQ}, label={lab2/lst/fluentInterfaceLinq}]
var translations = new Dictionary<string, string>
{
	{"cat", "chat"},
	{"dog", "chien"},
	{"fish", "poisson"},
	{"bird", "oiseau"}
};

// Find translations for English words containing the letter "a",
// sorted by length and displayed in uppercase
IEnumerable<string> query = translations
.Where(t => t.Key.Contains("a"))
.OrderBy(t => t.Value.Length)
.Select(t => t.Value.ToUpper());

// The same query constructed progressively:
var filtered = translations.Where(t => t.Key.Contains("a"));
var sorted  = filtered.OrderBy(t => t.Value.Length);
var finalQuery = sorted.Select(t => t.Value.ToUpper());
\end{lstlisting}

\subsubsection{Zadanie 2}

Utwórz projekt .NET 5.0. Dodaj do niego klasę \texttt{Car} i umieść do niej właściwości ze słowem kluczowym \texttt{init}\footnote{Słowo kluczowe \texttt{init} pojawiło się w C\# 9.0 i~stanowi ułatwienie w tworzeniu obiektów niemutowalnych (niezmiennych). Wcześniej konieczne było wykorzystywanie słowa kluczowego \texttt{readonly} i tworzenie obszernych konstruktorów.}\ref{lab2/lst/immutableProperties}.

% Jeśli użyjemy jedynie akcesora get przy właściwości to można przypisać jej wartość w konstruktorze, ale konieczne jest przekazanie wartości przez argument konstruktora :(

% OPOWIEDZIEĆ O REKORDACH
%Ponieważ w przypadku chęci utworzenia nowego obiektu niemutowalnego, który będzie posiadał inne wartości tylko kilku właściwości konieczne jest tworzenie nowego obiektu i ponowne przypisanie za pomoc listy inicjalizacji wszystkich właściwości pojawiły się w C\# również rekordy. Wykorzystując rekordy można przy przypisywaniu rekordu do innego rekordu użyć słowa kluczowego \texttt{with} i zmienić tylko tę jedną właściwość na której nam zależy, reszta zostanie automatycznie przepisana. Dodatkowo w przypadku rekordów porównywanie dwóch obiektów przebiega pole po polu. Kompilator automatycznie tworzy konstruktor kopiujący. Rekord wraz z~konstruktorem oraz destruktorem tworzy się za pomocą poniższego zapisu:
%\begin{lstlisting}
%	public record Friend(string FirstName,string MiddleName, string LastName);
%\end{lstlisting}
%Obiekt typu \texttt{Friend} będzie posiadał właściwości takie jak argumenty rekordu.
%Przykład stosowania rekordów:
%\begin{lstlisting}
%	using System;
%	namespace RecordsVS
%	{
%		public record Friend
%		{
%			public string FirstName { get; init; }
%			public string MiddleName { get; init; }   
%			public string LastName { get; init; }        
%		}
%		public record SuperFriend : Friend
%		{
%			public bool IsBestFriend { get; init; }
%		}
%		class Program
%		{
%			static void Main(string[] args)
%			{
%				var friendA = new Friend()
%				{
%					FirstName = "Chandler",
%					MiddleName = "Miruel",
%					LastName = "Bing"
%				};
%				var friendB = friendA with {LastName = "Geller"};
%				var friendC = friendB;
%				var superFriendA = new SuperFriend()
%				{
%					FirstName = "Ross",
%					LastName = "Geller",
%					IsBestFriend = true,
%				};
%				Friend friendD = superFriendA;
%				Console.WriteLine(friendB); //Friend { FirstName = Chandler, MiddleName = Miruel, LastName = Geller }
%				Console.WriteLine(friendA == friendB); //not equal
%				Console.WriteLine(friendB == friendC); //equal
%				
%				System.Console.WriteLine(friendD == superFriendA);//true
%			}
%		}
%	}
%\end{lstlisting}



\begin{lstlisting}[caption={Inicjalizacja obiektu klasy \texttt{Car} ze niemutowalnymi właściwościami}, label={lab2/lst/immutableProperties}]
public string Manual { get; init;}
public string Engine { get; init; }
public string Seats { get; init; }
public string TripComputer { get; init; }
public string Gps { get; init; }
\end{lstlisting}


Zdefiniuj w klasie \texttt{Car} prywatny konstruktor (wykorzystaj modyfikator dostępu private):
\begin{lstlisting}
private Car() { }
\end{lstlisting}
Zwróć uwagę, że nie będzie możliwe utworzenie obiektu klasy \texttt{Car} za pomocą słowa kluczowego \texttt{new} w powodu braku dostępnego konstruktora.

W osobnym pliku utwórz interfejs \texttt{ICarBuilder}, który będzie posiadał następujące metody zwracające typ \texttt{void}:
\begin{itemize}
	\item Reset(),
	\item SetEngine(),
	\item SetSeats(),
	\item SetTripComputer(),
	\item SetGps(),
	\item SetManual().
\end{itemize}

\textbf{Wewnątrz} klasy \texttt{Car}, zdefiniuj klasę \texttt{CarBuilder} implementującą interfejs \texttt{ICarBuilder}. Zwróć uwagę, że klasa ta ma dostęp do prywatnych składowych klas (w tym konstruktora). Utworzenie instancji tej klasy będzie możliwe jedynie przez klasę budowniczego. W klasie budowniczego utwórz prywatne pola typu \texttt{string}: \texttt{seats}, \texttt{engine}, \texttt{tripComputer} oraz \texttt{gps}.

Zaimplementuj interfejs \texttt{ICarBuilder}. W każdej metodzie ustawiającej dany fragment samochodu, przypisz jakiś łańcuch znaków pozwalający na późniejszą identyfikację np.:
\begin{lstlisting}
public class CarBuilder : ICarBuilder
{
	private string seats;
	//...
	
	public void SetSeats() => this.seats = "Four fancy seats";
	//...		
}
\end{lstlisting}

Do klasy \texttt{CarBuilder} dodaj metodę \texttt{Build} zwracającą typ \texttt{Car}. Utwórz w niej instancję klasy \texttt{Car}, przypisz do jej właściwości ustawione wcześniej wartości prywatnych pól np.:
\begin{lstlisting}
public Car Build()
{
	var car = new Car()
	{
		Engine = this.engine,
		//...
	};
	
	//...
}
\end{lstlisting}

W metodzie \texttt{Main} korzystając z budowniczego utwórz instancję klasy \texttt{Car}. Aby to zrobić stwórz obiekt \texttt{CarBuilder}, następnie wywołaj metody ustawiające poszczególne właściwości (nie jest konieczne wywoływanie ich wszystkich) i utwórz obiekt za pomocą metody \texttt{Build}. Sprawdź czy utworzona instancja typu \texttt{Car} zawiera prawidłowe wartości odpowiednich właściwości tej klasy. Możesz to zrobić wypisując je na ekranie konsoli. Ewentualnie można przeciążyć metodę \texttt{ToString} tak, aby zwracała łańcuch znaków zawierający te informacje.

Utwórz klasę kierownika \texttt{Director}. Niech posiada ona prywatne pole typu \texttt{ICarBuilder} oraz właściwość \texttt{set}:
\begin{lstlisting}
internal class Director()
{
	private ICarBuilder _builder;
	public ICarBuilder Builder { set { _builder = value; } }
	//...
}
\end{lstlisting}
Dodatkowo umieść w niej dwie metody: \texttt{BuildMinimalViableProduct} oraz \texttt{BuildFullFeaturedProduct}. Obie metody niech zwracają typ \texttt{void}. W pierwszej wywołaj dwie z dostępnych metod tworzenia obiektu przez budowniczego, w drugim wszystkie.
% Viable - wykonalny, realny

Z poziomu metody \texttt{Main} utwórz obiekt \texttt{CarBuilder} oraz \texttt{Director}. Przypisz kierownikowi obiekt budowniczego. Następnie wywołaj jedną z dostępnych metod kierownika. Na koniec pobierz ,,zbudowany'' obiekt za pomocą metody \texttt{Build} klasy \texttt{CarBuilder}. Sprawdź utworzone elementy obiektu \texttt{Car}.

\subsubsection{Zadanie 3}
Wewnątrz klasy \texttt{Car} utwórz drugiego budowniczego. Tym razem do implementacji wykorzystaj tzw. Płynne Interfejsy. Jedyną różnicą względem poprzedniej implementacji będzie fakt, że tym razem metody ustawiające wartości prywatnym polom (te metody, które ,,budują'' obiekt), będą zwracały instancję klasy budowniczego. Konieczne jest skorzystanie ze słowa kluczowego \texttt{this} zwracającego instancję kasy np.:
\begin{lstlisting}
public CarBuilderFluent SetSeats()
{
	this.seats = "Four fancy seats";
	return this;
}
\end{lstlisting}

Sprawdź działanie drugiej wersji wzorca projektowego Budowniczy. Tym razem kolejne metody można wywoływać z wykorzystaniem symbolu kropki.
\begin{lstlisting}
// Builder with a fluent interface
var builderFluent = new Car.CarBuilderFluent();
var carFluent = builderFluent.SetEngine()
	.SetGPS()
	.SetSeats()
	.SetTripComputer()
	.Build();
\end{lstlisting}